\documentclass{article}
\usepackage[margin=1in]{geometry}
\usepackage{mathtools, amsfonts, amsthm, graphicx, listings, xcolor}


\definecolor{codegreen}{rgb}{0,0.6,0}
\definecolor{codegray}{rgb}{0.5,0.5,0.5}
\definecolor{codepurple}{rgb}{0.58,0,0.82}
\definecolor{backcolour}{rgb}{0.95,0.95,0.92}

\lstdefinestyle{mystyle}{
    backgroundcolor=\color{backcolour},   
    commentstyle=\color{codegreen},
    keywordstyle=\color{magenta},
    numberstyle=\tiny\color{codegray},
    stringstyle=\color{codepurple},
    basicstyle=\ttfamily\footnotesize,
    breakatwhitespace=false,         
    breaklines=true,                 
    captionpos=b,                    
    keepspaces=true,                 
    numbers=left,                    
    numbersep=5pt,                  
    showspaces=false,                
    showstringspaces=false,
    showtabs=false,                  
    tabsize=2
}

\lstset{style=mystyle}

\title{Title}
\author{Sam Ly}

\begin{document}
\maketitle

\section*{Part A: By Hand (15 points)}

\textbf{Edges:} \((1,2), (1,3), (2,3), (3,4), (4,5), (4,6), (5,6), (5,7), (6,7), (7,8)\)

\noindent\textbf{Nodes:} \(\{1,2,3,4,5,6,7,8\}\)

\noindent \textbf{No. Edges:} 10

\begin{center}
    \includegraphics[width=0.3\textwidth]{figures/main_graph.png}
\end{center}

\begin{enumerate}
    \item {
        Density (3 points)
        \begin{itemize}
            \item {
                Write the formula for density.

                \[D = \frac{2E}{N(N-1)}\]
            }

            \item {
                Count nodes and edges.

                8 nodes, 10 edges.
            }

            \item {
                Compute the desnity of this graph.

                \[D = \frac{2(8)}{10(9)} = 0.3555\]
            }
        \end{itemize}
    }

    \item {
        Local Clustering Coefficient for Node 3 (3 points)

        \begin{itemize}
            \item {
                Write the formula.

                \[C_i = \frac{2E_i}{k_i(k_i-1)}\]
            }

            \item {
                Identify Node 3’s neighbors.

                Node 3's neighbors are node 1, node 2, and node 4.
            }

            \item {
                Count edges among them.

                There is one edge between node 3's neighbors: \((1,2)\).
            }

            \item {
                Compute \(C_3\).

                \[C_3 = \frac{2(1)}{3(2)} = \frac{1}{3}\]
            }
        \end{itemize}

        \item {
            Global Clustering Coefficient (3 points)

            \begin{itemize}
                \item {
                    Compute the local clustering coefficient for each node with 
                    degree \(\ge\) 2.

                    \begin{center}
                        \begin{tabular}{c c c c}
                            Node \(i\)  & \(k_i\)   & \(E_i\)   & \(C_i\)   \\
                            1           & 2         & 1         & 1         \\
                            2           & 2         & 1         & 1         \\
                            3           & 3         & 1         & \(\frac{1}{3}\)\\
                            4           & 3         & 1         & \(\frac{1}{3}\)\\
                            5           & 3         & 2         & \(\frac{2}{3}\)\\
                            6           & 3         & 2         & \(\frac{2}{3}\)\\
                            7           & 2         & 1         & 1         \\
                            8           & -         & -         & -         \\
                        \end{tabular}
                    \end{center}
                }

                \item {
                    Average them.

                    \[C = \frac{1}{7} \sum_{i=1}^{7}C_i = \frac{1 + 1 + \frac{1}{3} + \frac{1}{3} + \frac{2}{3} + \frac{2}{3} + 1}{7} = \frac{5}{7}\]
                }
                
                
            \end{itemize}

            \item {
                Average Path Length (4 points)

                \begin{itemize}
                    \item {
                        List all unique pairs of nodes.

                        \begin{description}
                            \item[Starting at 1] \((1,2), (1,3), (1,4), (1,5), (1,6), (1,7), (1,8) \)
                            \item[Starting at 2] \((2,3), (2,4), (2,5), (2,6), (2,7), (2,8) \)
                            \item[Starting at 3] \((3,4), (3,5), (3,6), (3,7), (3,8) \)
                            \item[Starting at 4] \((4,5), (4,6), (4,7), (4,8) \)
                            \item[Starting at 5] \((5,6), (5,7), (5,8) \)
                            \item[Starting at 6] \((6,7), (6,8) \)
                            \item[Starting at 7] \((7,8) \)
                        \end{description}
                    }

                    \item {
                        Find the shortest distance \(d(i, j)\) for each pair.

                        \begin{tabular}{c|*{8}{c}}
                                  & 1 & 2 & 3 & 4 & 5 & 6 & 7 & 8 \\ \hline
                                1 &   & 1 & 1 & 2 & 3 & 4 & 4 & 5 \\
                                2 &   &   & 1 & 2 & 3 & 4 & 4 & 5 \\
                                3 &   &   &   & 1 & 2 & 2 & 3 & 4 \\
                                4 &   &   &   &   & 1 & 1 & 2 & 3 \\
                                5 &   &   &   &   &   & 1 & 1 & 2 \\
                                6 &   &   &   &   &   &   & 1 & 2 \\
                                7 &   &   &   &   &   &   &   & 1 \\
                                8 &   &   &   &   &   &   &   &   \\
                        \end{tabular}
                    }

                    \item {
                        Compute the average. 
                        
                    }
                \end{itemize}
            }
        }
    }
\end{enumerate}
\end{document}