\documentclass{article}
\usepackage[margin=1in]{geometry}
\usepackage{mathtools, amsfonts, amsthm}

\newtheorem{theorem}{Theorem}
\newtheorem{lemma}{Lemma}
\newtheorem{proposition}{Proposition}

\theoremstyle{definition}
\newtheorem{definition}{Definition}
\newtheorem{example}{Example}

\title{Proofs Portfolio\\[5pt] \large MAT 3100W: Intro to Proofs}
\author{Sam Ly}

\begin{document}
\maketitle

\section{Introduction}
(Leave this blank for now. Here's an outline of course topics for your reference.)

\section{Mathematical concepts}

\section{Proof techniques}

\subsection{Direct Proofs}
Suppose \(a \equiv a' \pmod{m}\) and \(b \equiv b' \pmod{m}\). Prove the
following:

\begin{enumerate}
    \item {
        \(a + b \equiv a' + b' \pmod{m}\).

        \begin{proof}
            We begin with by defining \(a \equiv a' \pmod{m}\) as \(m \mid (a - a')\).
            Similarly, \(m \mid (b - b')\).

            Following from these definitions, we write:
            \begin{equation} \label{eq:1}
                a - a' = m \times k_1
            \end{equation}
            \begin{equation} \label{eq:2}
                b - b' = m \times k_2
            \end{equation}

            We can add equations \ref{eq:1} and \ref{eq:2} together to get 
            \(a + b - a' - b' = m \times k_1 + m \times k_2\).

            With some factoring, we get \((a + b) - (a' + b') = m (k_1 + k_2)\).

            By definition, we find that \(m \mid (a + b) - (a' + b')\), and thus
            \(a + b \equiv a' + b' \pmod{m}\).
        \end{proof}
    }

    \item {
        \(a - b \equiv a' - b' \pmod{m}\).

        \begin{proof}
            Following from Proof 1, we can instead subtract equation \ref{eq:1}
            and \ref{eq:2} to get 
            \\ \(a - b - a' + b' = m \times k_1 - m \times k_2\).

            With some factoring, we get \((a-b) - (a'-b') = m(k_1 - k_2)\).

            By definition, we find that \(m \mid (a - b) - (a' - b')\), and thus
            \(a - b \equiv a' - b' \pmod{m}\).
        \end{proof}
    }

    \item {
        \(a \times b \equiv a' \times b' \pmod{m}\).

        \begin{proof}
            Following from equation \ref{eq:1}, we get 
            \begin{equation} \label{eq:3}
                a = a' + m \times k_1.
            \end{equation}
            Similarly, from equation \ref{eq:2}, we  get 
            \begin{equation} \label{eq:4}
                b = b' + m \times k_2.
            \end{equation}

            By multiplying equations \ref{eq:3} and \ref{eq:4}, we get 
            \(a \times b = (a' + m\times k_1) (b' + m\times k_2)\).

            \textit{From now on, I will omit the \(\times\) symbol.}

            By distributing, we get 
            \[ab = a'b' + a'mk_2 + b'mk_1 + m^2k_1k_2.\]
            
            We can factor out \(m\) to find 
            \[ab = a'b' + m(a'k_2 + b'k_1 + mk_1k_2).\]

            We can subtract \(a'b'\) from both sides to find 
            \[ab - a'b' = m(a'k_2 + b'k_1 + mk_1k_2).\]

            By definition, we see that \(m \mid (ab - a'b')\), and, by extension, 
            \(ab \equiv a'b' \pmod{m}\).
                
        \end{proof}
    }
\end{enumerate}

\subsection{Proof by Induction}
As an example of Proof by Induction, we will prove the following.

\begin{proposition}
    Let \(F_n\) be the \(n\)-th Fibonacci number, where \(F_0 = F_1 = 1\)
    and \(F_n = F_{n-1} + F_{n-2}\). Prove that \(F_n \le 1.9^n\) for all
    \(n \ge 1\).
\end{proposition}
\begin{proof}
    We begin by verifying the relation for small \(n\) to create our 
    base cases:
    \[n = 1, F_1 = 1 \le 1.9^1 = 1.9\]
    \[n = 2, F_2 = 2 \le 1.9^2 = 3.61.\]

    We then form our inductive hypothesis by assumimg \(F_n \le 1.9^n\) for \(1 \le n \le k\).
    \[F_{k+1} \le 1.9^{k+1}\]
    \[F_k + F_{k-1} \le 1.9^{k+1}.\]

    Using our inductive hypothesis, we see that \(F_k \le 1.9^k\)
    and \(F_{k-1} \le 1.9^{k-1}\).

    So, \(F_k + F_{k-1} \le 1.9^k + 1.9^{k-1}\).

    By refactoring \(1.9^k + 1.9^{k-1}\), we get:
    \[1.9^k + 1.9^{k-1} = 1.9(1.9^{k-1}) + 1.9^{k-1} = 2.9(1.9^{k-1}).\]

    Also, \(1.9^{k+1}\) can be rewritten as \(1.9^2 (1.9^{k-1}) = 3.61(1.9^{k-1})\).

    Finally, we see
    \[F_{k+1} = F_k + F_{k-1} \le 2.9(1.9^{k-1}) \le 3.61(1.9^{k-1}) = 1.9^{k+1}\]
    \[F_{k+1} \le 1.9^{k+1}.\]

    Therefore, \(F_n \le 1.9^n\) for all \(n \ge 1\).
\end{proof}

\begin{proposition}
Look up the Tower of Hanoi puzzle. Prove that given a stack of 
disks, you can solve the puzzle in moves.

\end{proposition}
\begin{proof}
    We begin by defining the Tower of Hanoi problem. 
    
    In this problem, we begin 
    with a stack of \(n\) disks. The disks are ordered from largest at the bottom 
    to smallest at the top. We are also given 3 `spots' to place our disks under 
    one condition: that we never place a larger disk on top of a smaller disk. 

    Following these rules, what is the minimum number of moves required to move
    the entire pile to a new `spot'?
    
    We define the function \(f: \mathbb{N} \rightarrow \mathbb{N}\) such that it 
    maps the starting stack height \(n\) to the minimum number of moves required 
    to move the entire pile \(f(n)\).

    Before immediately proving that \(f(n) = 2^n - 1\), it is more intuitive to 
    first define \(f\) as a recurrence relation, then prove that the recurrence
    relation is equal to \(2^n -1\).

    We notice that moving the entire pile  of \(n\) disks essentially requires 3 `phases':
    \begin{enumerate}
        \item Moving the top \(n-1\) disks onto a single pile.
        \item Moving the \(n\)th disk to another vacant spot.
        \item Moving the top \(n-1\) disks onto the new spot.
    \end{enumerate}

    Thus, we know that \(f(n) = f(n-1) + 1 + f(n-1) = 1 + 2f(n-1)\), where \(f(1) = 1\).
    We can then prove \(f(n) = 2^n -1\) using induction. 

    We begin with our base cases:
    \begin{center}
        \begin{tabular}{c c}
            \(n\)   & \(f(n)\) \\
            \hline \\
            1       & \(1 = 2^1 - 1\)\\
            2       & \(3 = 2^2 - 1\) \\
            3       & \(7 = 2^3 - 1\) \\
        \end{tabular}
    \end{center}

    Now, we assume that \(f(k) = 2^k -1 \) for all \(1 \le k \le n\).
    
    We see that
    \[f(k+1) = 1 + 2f(k)\]
    \[f(k+1) = 1 + 2(2^k - 1)\]
    \[f(k+1) = 2^{k+1} - 1.\]

    Thus, \(f(n) = 2^n - 1\).
\end{proof}

\section{Final project}

\section{Conclusion and reflection}

\pagebreak
\appendix
\begin{center}
    \LARGE Appendix
\end{center}
\noindent (The first section, ``Course objectives and student learning outcomes'' is just here for your reference.)
\section{Course objectives and student learning outcomes}

\begin{enumerate}
    \item Students will learn to identify the logical structure of mathematical statements and apply appropriate strategies to prove those statements.
    \item Students learn methods of proof including direct and indirect proofs (contrapositive, contradiction) and induction.
    \item Students learn the basic structures of mathematics, including sets, functions, equivalence relations, and the basics of counting formulas.
    \item Students will be able to prove multiply quantified statements.
    \item Students will be exposed to well-known proofs, like the irrationality of $\sqrt{2}$ and the uncountability of the reals.
\end{enumerate}

\subsection{Expanded course description}
\begin{itemize}
    \item Propositional logic, truth tables, DeMorgan's Laws
    \item Sets, set operations, Venn diagrams, indexed collections of sets
    \item Conventions of writing proofs
    \item Proofs
    \begin{itemize}
        \item Direct proofs
        \item Contrapositive proofs
        \item Proof by cases
        \item Proof by contradiction
        \item Existence and Uniqueness proofs
        \item Proof by Induction
    \end{itemize}
    \item Quantifiers
    \begin{itemize}
        \item Proving universally and existentially quantified statements
        \item Disproving universally and existentially quantified statements
        \item Proving and disproving multiply quantified statements
    \end{itemize}
    \item Number systems and basic mathematical concepts
    \begin{itemize}
        \item The natural numbers and the integers, divisibility, and modular arithmetic
        \item Counting: combinations and permutations, factorials
        \item Rational numbers, the irrationality of $\sqrt{2}$
        \item Real numbers, absolute value, and inequalities
    \end{itemize}
    \item Relations and functions
    \begin{itemize}
        \item Relations, equivalence relations
        \item Functions
        \item Injections, surjections, bijections
    \end{itemize}
    \item Cardinality
    \begin{itemize}
        \item Countable and uncountable sets
        \item Countability of the rational numbers, $\mathbb{Q}$
        \item Uncountability of the real numbers, $\mathbb{R}$
    \end{itemize}
\end{itemize}
\end{document}
