\documentclass{article}
\usepackage[margin=1in]{geometry}
\usepackage{mathtools, amsfonts, amsthm, hyperref}
\usepackage{cleveref}

\newtheorem{theorem}{Theorem}
\newtheorem{lemma}{Lemma}
\newtheorem{proposition}{Proposition}

\theoremstyle{definition}
\newtheorem{definition}{Definition}
\newtheorem{example}{Example}

\title{Proofs Portfolio\\[5pt] \large MAT 3100W: Intro to Proofs}
\author{Sam Ly}

\begin{document}
\maketitle

\section{Introduction}
(Leave this blank for now. Here's an outline of course topics for your reference.)

\section{Mathematical concepts}

\subsection{Logic, truth tables, and DeMorgan's laws}

\subsubsection{Logical Statements}
\begin{definition}
    A logical statement is a statement that can either be \textbf{true} or \textbf{false}. 
    Logical statements must be unambiguous, meaning all ratioal agents with access 
    to the same information will come to the same conclusion. 
\end{definition}

\begin{example}
    ``The sun rose today.'' is a \textbf{true} logical statement.
\end{example}

\begin{proof}
    We begin by observing that the we can currently see the sun in the sky and 
    that we could not see the sun in the sky last night. If we can not see the 
    sun in the sky, it must be below the horizon. Because the sun follows a 
    continuous path, and it had been below the horizon last night, 
    it must have crossed the horizon at some point between last night and now. 
    Thus the sun must have risen today.
\end{proof}

\subsubsection{Truth Tables}
\begin{definition}
    Certain logical statements' \textbf{truth value} depends on the truth of 
    other statements. For example, ``the sun rose today \textbf{and} it rained 
    today'' requires both statements to be true in order for the overall statements 
    to be true. If the sun rose but it didn't rain, or if the sun hasn't risen 
    but it is raining, the overall statement is false. Thus, to visualize this 
    relationship, it is useful to have a table to lay out the possibilities. 
\end{definition}

\begin{example}
    \(A = \) the sun rose today, \(B = \) it rained today.
    \[
        \begin{tabular}{|c|c|c|}
        \hline
        A & B & $A \land B$ \\
        \hline
        T & T & T \\
        T & F & F \\
        F & T & F \\
        F & F & F \\
        \hline
        \end{tabular}
    \]
\end{example}

\subsubsection{DeMorgan's Laws}
\begin{definition}
    Logical statements and their combinations have their own form of algebra.  One of the fundamental rules are DeMorgan's Laws, which state how to find the complements of conjunctions and disjunctions.
\end{definition}

\begin{theorem}
    \label{prop:de-morgans-laws}
    DeMorgan's Laws:
    \begin{enumerate}
        \item \( \neg(A \land B) = \neg A \lor \neg B\)
        \item \( \neg(A \lor B) = \neg A \land \neg B\)
    \end{enumerate}
\end{theorem}


\subsection{Sets}

\begin{definition}
    Set: An unordered collection of unique elements. 
\end{definition}

\subsubsection{Unions, intersections, complements, and set differences}

\begin{definition}
    Union: the union of two sets \(A\), \(B\) is the set that contain elements that are 
    in \(A\), or in \(B\), or both. 
    \[A \cup B = \{x \mid x \in A \lor x \in B \}.\]
\end{definition}

\begin{definition}
    Intersection: the intersection of two sets \(A\), \(B\) is the set that contains 
    elements that are in both \(A\) and \(B\) at the same time.
    \[A \cap B = \{x \mid x \in A \land x \in B\}.\]
\end{definition}

\begin{definition}
    Difference: the set difference of two sets \(A\), \(B\) is the set that contains 
    all elements of \(A\) that are not in \(B\). This operation is not commutative.`'
    \[A \setminus B = \{x \mid x\in A \land x \not\in B\}.\]
\end{definition}

\begin{definition}
    Complement: the complement of a set \(A\) is the set of all elements that are 
    not in \(A\). For the complements of a set to be defined, it must be a subset 
    of the unversal set \(\mathcal{U}\). In other words, it is the set difference
    between \(\mathcal{U }\) and \(A\).
    \[A^c = \mathcal{U} \setminus A.\]
\end{definition}

\begin{theorem}
    DeMorgan's Laws for Sets:
    \begin{enumerate}
        \item \( (A \cap B)^c =  A^c \cup  B^c\)
        \item \( (A \cup B)^c =  A^c \cap  B^c\)
    \end{enumerate}

\end{theorem}

\subsubsection{Venn diagrams}

\begin{definition}
    Venn diagrams: a visual aid for understanding sets of objects and their 
    relationships.

    \begin{center}
        \includegraphics[width=0.6\textwidth]{./images/set-theory-venn.png}
    \end{center}
\end{definition}

\subsection{Numbers and number systems}

\begin{definition}
    Number: values that symbolize quantities. 
\end{definition}

\begin{definition}
    Number system: way of representing numbers. Some are more sophisticated than 
    others.
\end{definition}

\subsubsection{Parity, divisibility, and modular arithmetic}

\begin{definition}
    Divisibility: a number \(n \in \mathbb{Z}\) is divisible by another number 
    \(m\) if and only if \(n = k \times m\) for some integer \(k\). 
\end{definition}

\begin{definition}
    Parity: the property of a number being even or odd. The number is even if it
    is divisible by two, and odd otherwise. 
\end{definition}

\begin{definition}
    \label{def:modular-arithmetic}
    Modular arithmetic: a number system that groups numbers into equivalence 
    classes based on their remainder when divided by a specific integer. 

    More formally, for integers \(n \), \(r \), and \(m \), we say \(n \) is 
    \textbf{congruent} to \(r \) modulo \(m \) if \((n-r)\) is divisible by \(m\).
    \[n \equiv r \pmod{m} \Leftrightarrow m \mid (n - r)\]

    For example, \(5 \equiv 11 \pmod3\) since they both have a remainder 2 when 
    divided by 3, and because \(11 - 5 = 6\) is divisible by 3.

    Standard arithmetic operations \(+, -, \text{and } \times \) are well-defined
    under modular arithmetic. However, \(\div\) is not always well defined. These 
    operations work the same way as they do in standard arithmetic.

    Notice that the parity of a number is equivalent to its divisability by 2, and 
    a number's divisibility by \(m \in \mathbb{N} > 0\) is a equivalent to it being 
    congruent to \(0\) modulo \(m\). 
\end{definition}

\begin{proposition}
    \label{prop:modular-arithmetic1}
    If \(a \equiv b \pmod m\), 
    then \(b \equiv a \pmod m \).
\end{proposition}

\begin{proposition}
    \label{prop:modular-arithmetic2}
    If \(a \equiv a' \pmod m \) and \(b \equiv b' \pmod m\), then: 
    \begin{enumerate}
        \item \(a + b \equiv a' + b' \pmod m \)
        \item \(a - b \equiv a' - b' \pmod m \)
        \item \(a \times b \equiv a' \times b' \pmod m \)
    \end{enumerate}
\end{proposition}


\subsubsection{Rational and irrational numbers}

\begin{definition}
    Rational numbers \(\mathbb{Q}\): the set of numbers that can be expressed 
    as a ratio of two integers.
\end{definition}

\begin{proposition}
    \(\mathbb{Q }\) is countably infinite. 
\end{proposition}

\begin{definition}
    Irrational numbers: the set of numbers that can't be expressed as a ratio 
    of two integers. 
\end{definition}

\begin{proposition}
    The set of all irrational numbers is not countably infinite.
\end{proposition}

\subsubsection{Real numbers, absolute value, and inequalities}

\begin{definition}
    Real numbers \(\mathbb{R}\): the set of all numbers on our number line.
\end{definition}

\begin{proposition}
    \(\mathbb{R }\) is not countably infinite.
\end{proposition}

\subsubsection{Combinatorics: combinations, permutations, and factorials.}

\begin{definition}
    Combinations \(C(n, r)\): the cardinality of the set of all subsets of a specific cardinality.
\end{definition}

\begin{definition}
    Permutations \(P(n, r)\): the cardinality of the set of all orderings of a specific length.
\end{definition}

\begin{definition}
    Factorial: the product of natural numbers before it down to zero.
    \[5! = 5 \times 4 \times 3 \times 2 \times 1 .\]
\end{definition}

\subsubsection{Countable sets}
\begin{definition}
    Countable set: a set that is either finite, or that has a bijection to the 
    natural numbers. A set is countably infinite if it has a bijection to the 
    natural numbers.
\end{definition}
\subsubsection{Uncountable sets}
\begin{definition}
    Uncountable set: a set that is infinite and there does not exists a bijection 
    from it to the natural numbers.
\end{definition}

\subsection{Relations and functions}

\subsubsection{Relations and equivalence relations}

\begin{definition}
    Relation \(R\): a set of ordered pairs that represents if a two elemeent \(a, b \in S\) 
    are related. \(a \text{and} b\) are related if and only if \((a,b) \in R\).
\end{definition}

\begin{definition}
    Equivalence relations: a special type of relation on a set that satisfies 
    the properties of being symmetric, reflexive, and transitive.
\end{definition}

\subsubsection{Functions}

\begin{definition}
    Function: a mapping from a set called the domain to elemeents in a set called 
    the codomain. 
\end{definition}


\subsubsection{Injections (one-to-one), surjections (onto), and bijections}

\section{Proof techniques}

\subsection{Direct Proofs}
\begin{definition}
    Direct proof: using fundamental rules of logic to prove a statement.
\end{definition}



Using the properties of modular arithmetic in  \cref{def:modular-arithmetic},
suppose \(a \equiv a' \pmod{m}\) and \(b \equiv b' \pmod{m}\). Prove the
following: 

\begin{enumerate}
    \item {
        \(a + b \equiv a' + b' \pmod{m}\).

        \begin{proof}
            We begin with by defining \(a \equiv a' \pmod{m}\) as \(m \mid (a - a')\).
            Similarly, \(m \mid (b - b')\).

            Following from these definitions, we write:
            \begin{equation} \label{eq:1}
                a - a' = m \times k_1
            \end{equation}
            \begin{equation} \label{eq:2}
                b - b' = m \times k_2
            \end{equation}

            We can add equations \cref{eq:1} and \cref{eq:2} together to get 
            \(a + b - a' - b' = m \times k_1 + m \times k_2\).

            With some factoring, we get \((a + b) - (a' + b') = m (k_1 + k_2)\).

            By definition, we find that \(m \mid (a + b) - (a' + b')\), and thus
            \(a + b \equiv a' + b' \pmod{m}\).
        \end{proof}
    }

    \item {
        \(a - b \equiv a' - b' \pmod{m}\).

        \begin{proof}
            Following from Proof 1, we can instead subtract equation \cref{eq:1}
            and \cref{eq:2} to get 
            \\ \(a - b - a' + b' = m \times k_1 - m \times k_2\).

            With some factoring, we get \((a-b) - (a'-b') = m(k_1 - k_2)\).

            By definition, we find that \(m \mid (a - b) - (a' - b')\), and thus
            \(a - b \equiv a' - b' \pmod{m}\).
        \end{proof}
    }

    \item {
        \(a \times b \equiv a' \times b' \pmod{m}\).

        \begin{proof}
            Following from equation \cref{eq:1}, we get 
            \begin{equation} \label{eq:3}
                a = a' + m \times k_1.
            \end{equation}
            Similarly, from equation \cref{eq:2}, we  get 
            \begin{equation} \label{eq:4}
                b = b' + m \times k_2.
            \end{equation}

            By multiplying equations \cref{eq:3} and \cref{eq:4}, we get 
            \(a \times b = (a' + m\times k_1) (b' + m\times k_2)\).

            \textit{From now on, I will omit the \(\times\) symbol.}

            By distributing, we get 
            \[ab = a'b' + a'mk_2 + b'mk_1 + m^2k_1k_2.\]
            
            We can factor out \(m\) to find 
            \[ab = a'b' + m(a'k_2 + b'k_1 + mk_1k_2).\]

            We can subtract \(a'b'\) from both sides to find 
            \[ab - a'b' = m(a'k_2 + b'k_1 + mk_1k_2).\]

            By definition, we see that \(m \mid (ab - a'b')\), and, by extension, 
            \(ab \equiv a'b' \pmod{m}\).
                
        \end{proof}
    }
\end{enumerate}

\subsection{Transformation of conditionals}

\begin{definition}
    Transformation of conditionals: using rules of conditional logic to prove 
    conditional statements. 
\end{definition}

\subsubsection{Inverse statements}

\subsubsection{Converse statements}

\subsubsection{Contrapositive proofs}

\subsubsection{Bidirectional ("if and only if" proofs)}

\subsection{Quantifiers}

\begin{definition}
    Quantifier: a logical expression that denotes whether a statement is true 
    for all cases or for specific cases. 
\end{definition}

\subsubsection{Universal quantifiers}

\subsubsection{Existential quantifiers}

\subsubsection{Multiply quantified statements}

\subsection{Existence and uniqueness proofs}

\begin{definition}
    Existence and uniqueness proof: a proof that results in us being sure that 
    an element exists with a given property, and that it is the only element that 
    exhibits such property. 
\end{definition}

\subsection{Proof by Induction}

\begin{definition}
    Proof by Induction: proof technique used to prove a statement is true for  
    a countably infinite set of discrete elements. 
\end{definition}

As an example of Proof by Induction, we will prove the following.

\begin{proposition}
        Show that every positive integer is a sum of one or more numbers of the 
        form \(2^r3^s\), where \(r\) and \(s\) are nonnegative integers and no 
        summand divides another. (For example, \(23 = 9 + 8 + 6\)).
\end{proposition}
\begin{proof}
    We proceed by mathematical induction. 

    We start with \(n = 0, 1\) as our base cases. We see that \(n = 0\) is true, 
    and \(n = 1\) is true because \(1 = 2^0 3^0\).
    
    Now, we create the inductive hypothesis that all nonnegative 
    integers strictly less than \(n\) have such summation.

    If \(n\) is even, we can construct a valid summation by noticing 
    that, from our inductive hypothesis, \(\frac{n}{2}\) has a valid summation 
    \(\sum_{i=1}^{k}2^{r_i}3^{s_i}\). Since none of these summands divide 
    any other summand, multiplying all summands by 2 also creates a set 
    of summands such that no summand divides another.

    If \(n\) is odd, we can also construct a valid summation by picking 
    a value \(3^t\) that is the biggest power of 3 that is less than or 
    equal to \(n\). Our proposition is trivially true if \(n = 3^t\). 
    Otherwise, we must find a value \(m = n - 3^t\).

    Since \(n\) and \(3^t\) are both odd, \(m\) must be even. Also notice 
    that \(m < n\). Thus, there must exist a valid summation
    \(m = \sum_{j=1}^{k}2^{r_j}3^{s_j}\) where all \(r_j \ge 1\).

    Since all summands of \(m\) are even, \(3^t\) can not be divisible 
    by any of the summands of \(m\). Also, since \(r_j \ge 1\), there 
    must not be any summand where \(s_j \ge t\) because if such summand 
    existed, we would find at least a value of 
    \(n = 3^t + 2(3^t) = 3^{t+1}\). This is a contradiction, since we 
    defined \(3^t\) as the largest power of 3 less than or equal to \(n\).

    Thus, \(n = \sum 2^r3^s\) where no summand divides another for all 
    nonnegative integers \(n\).
\end{proof}

\begin{proposition}
    Let \(F_n\) be the \(n\)-th Fibonacci number, where \(F_0 = F_1 = 1\)
    and \(F_n = F_{n-1} + F_{n-2}\). Prove that \(F_n \le 1.9^n\) for all
    \(n \ge 1\).
\end{proposition}
\begin{proof}
    We proceed by induction, starting with the base cases, where \(n=1,2\):
    \[n = 1, F_1 = 1 \le 1.9^1 = 1.9\]
    \[n = 2, F_2 = 2 \le 1.9^2 = 3.61.\]

    We assume as an inductive hypothesis that \(F_n \le 1.9^n\) for \(1 \le n \le k\).

    Using our inductive hypothesis, we see that \(F_k \le 1.9^k\)
    and \(F_{k-1} \le 1.9^{k-1}\).
    % \[F_{k+1} \le 1.9^{k+1}\]
    % \[F_k + F_{k-1} \le 1.9^{k+1}.\]

    So, \(F_k + F_{k-1} \le 1.9^k + 1.9^{k-1}\).

    By refactoring \(1.9^k + 1.9^{k-1}\), we get:
    \[1.9^k + 1.9^{k-1} = 1.9(1.9^{k-1}) + 1.9^{k-1} = 2.9(1.9^{k-1}).\]

    Also, \(1.9^{k+1}\) can be rewritten as \(1.9^2 (1.9^{k-1}) = 3.61(1.9^{k-1})\).

    Finally, we see
    \[F_{k+1} = F_k + F_{k-1} \le 2.9(1.9^{k-1}) \le 3.61(1.9^{k-1}) = 1.9^{k+1}\]
    \[F_{k+1} \le 1.9^{k+1}.\]

    Thus, we conclude that by the principle of mathematical induction, 
    \(F_n \le 1.9^n\) for all \(n \ge 1\).
\end{proof}

\begin{proposition}
Look up the Tower of Hanoi puzzle. Prove that given a stack of 
disks, you can solve the puzzle in moves.

\end{proposition}
\begin{proof}
    We begin by defining the Tower of Hanoi problem. 
    
    In this problem, we begin 
    with a stack of \(n\) disks. The disks are ordered from largest at the bottom 
    to smallest at the top. We are also given 3 `spots' to place our disks under 
    one condition: that we never place a larger disk on top of a smaller disk. 

    Following these rules, what is the minimum number of moves required to move
    the entire pile to a new `spot'?
    
    We define the function \(f: \mathbb{N} \rightarrow \mathbb{N}\) such that it 
    maps the starting stack height \(n\) to the minimum number of moves required 
    to move the entire pile \(f(n)\).

    Before immediately proving that \(f(n) = 2^n - 1\), it is more intuitive to 
    first define \(f\) as a recurrence relation, then prove that the recurrence
    relation is equal to \(2^n -1\).

    We notice that moving the entire pile  of \(n\) disks essentially requires 3 `phases':
    \begin{enumerate}
        \item Moving the top \(n-1\) disks onto a single pile.
        \item Moving the \(n\)th disk to another vacant spot.
        \item Moving the top \(n-1\) disks onto the new spot.
    \end{enumerate}

    Thus, we know that \(f(n) = f(n-1) + 1 + f(n-1) = 1 + 2f(n-1)\), where \(f(1) = 1\).
    We can then prove \(f(n) = 2^n -1\) using induction. 

    We begin with our base cases:
    \begin{center}
        \begin{tabular}{c c}
            \(n\)   & \(f(n)\) \\
            \hline \\
            1       & \(1 = 2^1 - 1\)\\
            2       & \(3 = 2^2 - 1\) \\
            3       & \(7 = 2^3 - 1\) \\
        \end{tabular}
    \end{center}

    Now, we assume that \(f(k) = 2^k -1 \) for all \(1 \le k \le n\).
    
    We see that
    \[f(k+1) = 1 + 2f(k)\]
    \[f(k+1) = 1 + 2(2^k - 1)\]
    \[f(k+1) = 2^{k+1} - 1.\]

    Thus, \(f(n) = 2^n - 1\).
\end{proof}

\subsection{Proof by Contradiction}

\begin{definition}
    Proof by contradiction: proof technique that assumes the opposite of 
    our proposition, then showing that this leads to an absurd conclusion,
    ie. a contradiction. Used as a ``last resort'' proof technique. 
\end{definition}

\section{Final project}

\section{Conclusion and reflection}

\pagebreak
\appendix
\begin{center}
    \LARGE Appendix
\end{center}
\noindent (The first section, ``Course objectives and student learning outcomes'' is just here for your reference.)
\section{Course objectives and student learning outcomes}

\begin{enumerate}
    \item Students will learn to identify the logical structure of mathematical statements and apply appropriate strategies to prove those statements.
    \item Students learn methods of proof including direct and indirect proofs (contrapositive, contradiction) and induction.
    \item Students learn the basic structures of mathematics, including sets, functions, equivalence relations, and the basics of counting formulas.
    \item Students will be able to prove multiply quantified statements.
    \item Students will be exposed to well-known proofs, like the irrationality of $\sqrt{2}$ and the uncountability of the reals.
\end{enumerate}

\subsection{Expanded course description}
\begin{itemize}
    \item Propositional logic, truth tables, DeMorgan's Laws
    \item Sets, set operations, Venn diagrams, indexed collections of sets
    \item Conventions of writing proofs
    \item Proofs
    \begin{itemize}
        \item Direct proofs
        \item Contrapositive proofs
        \item Proof by cases
        \item Proof by contradiction
        \item Existence and Uniqueness proofs
        \item Proof by Induction
    \end{itemize}
    \item Quantifiers
    \begin{itemize}
        \item Proving universally and existentially quantified statements
        \item Disproving universally and existentially quantified statements
        \item Proving and disproving multiply quantified statements
    \end{itemize}
    \item Number systems and basic mathematical concepts
    \begin{itemize}
        \item The natural numbers and the integers, divisibility, and modular arithmetic
        \item Counting: combinations and permutations, factorials
        \item Rational numbers, the irrationality of $\sqrt{2}$
        \item Real numbers, absolute value, and inequalities
    \end{itemize}
    \item Relations and functions
    \begin{itemize}
        \item Relations, equivalence relations
        \item Functions
        \item Injections, surjections, bijections
    \end{itemize}
    \item Cardinality
    \begin{itemize}
        \item Countable and uncountable sets
        \item Countability of the rational numbers, $\mathbb{Q}$
        \item Uncountability of the real numbers, $\mathbb{R}$
    \end{itemize}
\end{itemize}
\end{document}
