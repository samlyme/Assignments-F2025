\documentclass{article}
\usepackage[margin=1in]{geometry}
\usepackage{mathtools, amsfonts, amsthm}

\newtheorem{theorem}{Theorem}
\newtheorem{lemma}{Lemma}
\newtheorem{proposition}{Proposition}

\theoremstyle{definition}
\newtheorem{definition}{Definition}
\newtheorem{example}{Example}

\title{Proofs Portfolio\\[5pt] \large MAT 3100W: Intro to Proofs}
\author{Sam Ly}

\begin{document}
\maketitle

\section{Introduction}
(Leave this blank for now. Here's an outline of course topics for your reference.)

\section{Proof techniques}
(Here we give examples of some proof techniques.)

\subsection{Proof by Induction}
As an example of Proof by Induction, we will prove the following.

\begin{proposition}
    Let \(F_n\) be the \(n\)-th Fibonacci number, where \(F_0 = F_1 = 1\)
    and \(F_n = F_{n-1} + F_{n-2}\). Prove that \(F_n \le 1.9^n\) for all
    \(n \ge 1\).
\end{proposition}
\begin{proof}
    We begin by verifying the relation for small \(n\) to create our 
    base cases:
    \[n = 1, F_1 = 1 \le 1.9^1 = 1.9\]
    \[n = 2, F_2 = 2 \le 1.9^2 = 3.61.\]

    We then form our inductive hypothesis by assumimg \(F_n \le 1.9^n\) for \(1 \le n \le k\).
    \[F_{k+1} \le 1.9^{k+1}\]
    \[F_k + F_{k-1} \le 1.9^{k+1}.\]

    Using our inductive hypothesis, we see that \(F_k \le 1.9^k\)
    and \(F_{k-1} \le 1.9^{k-1}\).

    So, \(F_k + F_{k-1} \le 1.9^k + 1.9^{k-1}\).

    By refactoring \(1.9^k + 1.9^{k-1}\), we get:
    \[1.9^k + 1.9^{k-1} = 1.9(1.9^{k-1}) + 1.9^{k-1} = 2.9(1.9^{k-1}).\]

    Also, \(1.9^{k+1}\) can be rewritten as \(1.9^2 (1.9^{k-1}) = 3.61(1.9^{k-1})\).

    Finally, we see
    \[F_{k+1} = F_k + F_{k-1} \le 2.9(1.9^{k-1}) \le 3.61(1.9^{k-1}) = 1.9^{k+1}\]
    \[F_{k+1} \le 1.9^{k+1}.\]

    Therefore, \(F_n \le 1.9^n\) for all \(n \ge 1\).
\end{proof}

\pagebreak
\appendix
\begin{center}
    \LARGE Appendix
\end{center}
\noindent (The first section, ``Course objectives and student learning outcomes'' is just here for your reference.)
\section{Course objectives and student learning outcomes}

\begin{enumerate}
    \item Students will learn to identify the logical structure of mathematical statements and apply appropriate strategies to prove those statements.
    \item Students learn methods of proof including direct and indirect proofs (contrapositive, contradiction) and induction.
    \item Students learn the basic structures of mathematics, including sets, functions, equivalence relations, and the basics of counting formulas.
    \item Students will be able to prove multiply quantified statements.
    \item Students will be exposed to well-known proofs, like the irrationality of $\sqrt{2}$ and the uncountability of the reals.
\end{enumerate}

\subsection{Expanded course description}
\begin{itemize}
    \item Propositional logic, truth tables, DeMorgan's Laws
    \item Sets, set operations, Venn diagrams, indexed collections of sets
    \item Conventions of writing proofs
    \item Proofs
    \begin{itemize}
        \item Direct proofs
        \item Contrapositive proofs
        \item Proof by cases
        \item Proof by contradiction
        \item Existence and Uniqueness proofs
        \item Proof by Induction
    \end{itemize}
    \item Quantifiers
    \begin{itemize}
        \item Proving universally and existentially quantified statements
        \item Disproving universally and existentially quantified statements
        \item Proving and disproving multiply quantified statements
    \end{itemize}
    \item Number systems and basic mathematical concepts
    \begin{itemize}
        \item The natural numbers and the integers, divisibility, and modular arithmetic
        \item Counting: combinations and permutations, factorials
        \item Rational numbers, the irrationality of $\sqrt{2}$
        \item Real numbers, absolute value, and inequalities
    \end{itemize}
    \item Relations and functions
    \begin{itemize}
        \item Relations, equivalence relations
        \item Functions
        \item Injections, surjections, bijections
    \end{itemize}
    \item Cardinality
    \begin{itemize}
        \item Countable and uncountable sets
        \item Countability of the rational numbers, $\mathbb{Q}$
        \item Uncountability of the real numbers, $\mathbb{R}$
    \end{itemize}
\end{itemize}
\end{document}
