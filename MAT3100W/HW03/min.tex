\documentclass{article}
\usepackage[margin=1in]{geometry}
\usepackage{mathtools, amsfonts, amsthm, graphicx, listings, xcolor}


\definecolor{codegreen}{rgb}{0,0.6,0}
\definecolor{codegray}{rgb}{0.5,0.5,0.5}
\definecolor{codepurple}{rgb}{0.58,0,0.82}
\definecolor{backcolour}{rgb}{0.95,0.95,0.92}

\lstdefinestyle{mystyle}{
    backgroundcolor=\color{backcolour},   
    commentstyle=\color{codegreen},
    keywordstyle=\color{magenta},
    numberstyle=\tiny\color{codegray},
    stringstyle=\color{codepurple},
    basicstyle=\ttfamily\footnotesize,
    breakatwhitespace=false,         
    breaklines=true,                 
    captionpos=b,                    
    keepspaces=true,                 
    numbers=left,                    
    numbersep=5pt,                  
    showspaces=false,                
    showstringspaces=false,
    showtabs=false,                  
    tabsize=2
}

\lstset{style=mystyle}

\title{HW03}
\author{Sam Ly}

\begin{document}
\maketitle

\section*{Total Points: 20}

\section*{Required Exercise 1 [2]}

\begin{enumerate}
    \item {
        Type \verb|\(726 \equiv 23 \pmod{19}\)| to get 
        \(726 \equiv 23 \pmod{19}\).
    }
    \item {
        Prove or disprove that 
        \(726 \equiv 23 \pmod{19}\).

        \begin{proof}
            We start by stating the definition. \(726 \equiv 23 \pmod{19}\) is 
            the same as saying \(19 \mid (726 - 23)\).

            Simplifying the expression, we get \(19 \mid (703)\).

            We see that \(19 \times 37 = 703\). Therefore, \(726 \equiv 23 \pmod{19}\).
        \end{proof}
    }
\end{enumerate}

\section*{Required Exercise 2 [4]}

Suppose \(a \equiv a' \pmod{m}\) and \(b \equiv b' \pmod{m}\). Prove the
following:

\begin{enumerate}
    \item {
        \(a + b \equiv a' + b' \pmod{m}\).

        \begin{proof}
            We begin with by defining \(a \equiv a' \pmod{m}\) as \(m \mid (a - a')\).
            Similarly, \(m \mid (b - b')\).

            Following from these definitions, we write:
            \begin{equation} \label{eq:1}
                a - a' = m \times k_1
            \end{equation}
            \begin{equation} \label{eq:2}
                b - b' = m \times k_2
            \end{equation}

            We can add equations \ref{eq:1} and \ref{eq:2} together to get 
            \(a + b - a' - b' = m \times k_1 + m \times k_2\).

            With some factoring, we get \((a + b) - (a' + b') = m (k_1 + k_2)\).

            By definition, we find that \(m \mid (a + b) - (a' + b')\), and thus
            \(a + b \equiv a' + b' \pmod{m}\).
        \end{proof}
    }

    \item {
        \(a - b \equiv a' - b' \pmod{m}\).

        \begin{proof}
            Following from Proof 1, we can instead subtract equation \ref{eq:1}
            and \ref{eq:2} to get 
            \\ \(a - b - a' + b' = m \times k_1 - m \times k_2\).

            With some factoring, we get \((a-b) - (a'-b') = m(k_1 - k_2)\).

            By definition, we find that \(m \mid (a - b) - (a' - b')\), and thus
            \(a - b \equiv a' - b' \pmod{m}\).
        \end{proof}
    }

    \item {
        \(a \times b \equiv a' \times b' \pmod{m}\).

        \begin{proof}
            Following from equation \ref{eq:1}, we get 
            \begin{equation} \label{eq:3}
                a = a' + m \times k_1.
            \end{equation}
            Similarly, from equation \ref{eq:2}, we  get 
            \begin{equation} \label{eq:4}
                b = b' + m \times k_2.
            \end{equation}

            By multiplying equations \ref{eq:3} and \ref{eq:4}, we get 
            \(a \times b = (a' + m\times k_1) (b' + m\times k_2)\).

            \textit{From now on, I will omit the \(\times\) symbol.}

            By distributing, we get 
            \[ab = a'b' + a'mk_2 + b'mk_1 + m^2k_1k_2.\]
            
            We can factor out \(m\) to find 
            \[ab = a'b' + m(a'k_2 + b'k_1 + mk_1k_2).\]

            We can subtract \(a'b'\) from both sides to find 
            \[ab - a'b' = m(a'k_2 + b'k_1 + mk_1k_2).\]

            By definition, we see that \(m \mid (ab - a'b')\), and, by extension, 
            \(ab \equiv a'b' \pmod{m}\).
                
        \end{proof}
    }
\end{enumerate}

\section*{Required Exercise 3 [4]}

\begin{description}
    \item[Problem 8.2] {
        Give two reaons why `\(f: \mathbb{R} \rightarrow \mathbb{R}\) where 
        \(f(x) = \pm \sqrt{x}\)' is not a function.

        \begin{enumerate}
            \item {
                The first reason that \(f\) is not a function is because of an invalid 
                domain. The function \textit{should} map all reals (domain) to a subset 
                of the reals (codomain), but negative values actually map to imaginary 
                numbers.
            }

            \item {
                The second reason that \(f\) is not a function is because of the 
                `non-unique' output. One of the requirements for a function is that 
                one value from the domain maps to one and only one value from the 
                codomain. However, \(f\) maps a single value \(x\) to two values due to 
                the \(\pm\). 
            }
        \end{enumerate}
    }
\end{description}

\section*{Choice Exercise 4 [4]}

\begin{enumerate}
    \item[2.] {
        Show that every positive integer is a sum of one or more numbers of the 
        form \(2^r3^s\), where \(r\) and \(s\) are nonnegative integers and no 
        summand divides another. (For example, \(23 = 9 + 8 + 6\)).

        \begin{proof}
            We begin by showing that such summations exist for small \(n\). This 
            will act as a base case for an inductive step later.

            \begin{description}
                \item[\(n=0\)] Trivially true.
                \item[\(n=1\)] \(1 = 2^0 3^0\).
            \end{description}

            Now, we create the inductive hypothesis that all nonnegative 
            integers strictly less than \(n\) have such summation.

            If \(n\) is even, we can construct a valid summation by noticing 
            that, from our inductive hypothesis, \(\frac{n}{2}\) has a valid summation 
            \(\sum_{i=1}^{k}2^{r_i}3^{s_i}\). Since none of these summands divide 
            any other summand, multiplying all summands by 2 also creates a set 
            of summands such that no summand divides another.

            If \(n\) is odd, we can also construct a valid summation by picking 
            a value \(3^t\) that is the biggest power of 3 that is less than or 
            equal to \(n\). Our proposition is trivially true if \(n = 3^t\). 
            Otherwise, we must find a value \(m = n - 3^t\).

            Since \(n\) and \(3^t\) are both odd, \(m\) must be even. Also notice 
            that \(m < n\). Thus, there must exist a valid summation
            \(m = \sum_{j=1}^{k}2^{r_j}3^{s_j}\) where all \(r_j \ge 1\).

            Since all summands of \(m\) are even, \(3^t\) can not be divisible 
            by any of the summands of \(m\). Also, since \(r_j \ge 1\), there 
            must not be any summand where \(s_j \ge t\) because if such summand 
            existed, we would find at least a value of 
            \(n = 3^t + 2(3^t) = 3^{t+1}\). This is a contradiction, since we 
            defined \(3^t\) as the largest power of 3 less than or equal to \(n\).

            Thus, \(n = \sum 2^r3^s\) where no summand divides another for all 
            nonnegative integers \(n\).
        \end{proof}
    }
\end{enumerate}

\section*{Choice Exercise 8 [6]}

Look up the Tower of Hanoi puzzle. Prove that given a stack of \(n\) disks, you 
can solve the puzzle in \(2^n - 1\) moves.

\begin{proof}
    We begin by defining the Tower of Hanoi problem. 
    
    In this problem, we begin 
    with a stack of \(n\) disks. The disks are ordered from largest at the bottom 
    to smallest at the top. We are also given 3 `spots' to place our disks under 
    one condition: that we never place a larger disk on top of a smaller disk. 

    Following these rules, what is the minimum number of moves required to move
    the entire pile to a new `spot'?
    
    We define the function \(f: \mathbb{N} \rightarrow \mathbb{N}\) such that it 
    maps the starting stack height \(n\) to the minimum number of moves required 
    to move the entire pile \(f(n)\).

    Before immediately proving that \(f(n) = 2^n - 1\), it is more intuitive to 
    first define \(f\) as a recurrence relation, then prove that the recurrence
    relation is equal to \(2^n -1\).

    We notice that moving the entire pile  of \(n\) disks essentially requires 3 `phases':
    \begin{enumerate}
        \item Moving the top \(n-1\) disks onto a single pile.
        \item Moving the \(n\)th disk to another vacant spot.
        \item Moving the top \(n-1\) disks onto the new spot.
    \end{enumerate}

    Thus, we know that \(f(n) = f(n-1) + 1 + f(n-1) = 1 + 2f(n-1)\), where \(f(1) = 1\).
    We can then prove \(f(n) = 2^n -1\) using induction. 

    We begin with our base cases:
    \begin{center}
        \begin{tabular}{c c}
            \(n\)   & \(f(n)\) \\
            1       & \(1 = 2^1 - 1\)\\
            2       & \(3 = 2^2 - 1\) \\
            3       & \(7 = 2^3 - 1\) \\
        \end{tabular}
    \end{center}

    Now, we assume that \(f(k) = 2^k -1 \) for all \(1 \le k \le n\).
    
    We see that
    \[f(k+1) = 1 + 2f(k)\]
    \[f(k+1) = 1 + 2(2^k - 1)\]
    \[f(k+1) = 2^{k+1} - 1.\]

    Thus, \(f(n) = 2^n - 1\).
\end{proof}

\end{document}