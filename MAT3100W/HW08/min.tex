\documentclass{article}
\usepackage[margin=1in]{geometry}
\usepackage{mathtools, amsfonts, amsthm, amssymb, graphicx, listings, xcolor, pdfpages}


\definecolor{codegreen}{rgb}{0,0.6,0}
\definecolor{codegray}{rgb}{0.5,0.5,0.5}
\definecolor{codepurple}{rgb}{0.58,0,0.82}
\definecolor{backcolour}{rgb}{0.95,0.95,0.92}

\lstdefinestyle{mystyle}{
    backgroundcolor=\color{backcolour},   
    commentstyle=\color{codegreen},
    keywordstyle=\color{magenta},
    numberstyle=\tiny\color{codegray},
    stringstyle=\color{codepurple},
    basicstyle=\ttfamily\footnotesize,
    breakatwhitespace=false,         
    breaklines=true,                 
    captionpos=b,                    
    keepspaces=true,                 
    numbers=left,                    
    numbersep=5pt,                  
    showspaces=false,                
    showstringspaces=false,
    showtabs=false,                  
    tabsize=2
}

\lstset{style=mystyle}

\title{HW08}
\author{Sam Ly}

\begin{document}
\maketitle

\section*{Total points = 18}

\section*{Required Exercise 1 [4]}

\begin{enumerate}
    \item Done. See below.
    \item Add definitions for all subsections of section 2. 
    \item DNF.
    \item DNF.
    \item Done. See below. 
\end{enumerate}

\section*{Required Exercise 2 [2]}

Suppose that \(A_1, A_2, \dots, A_{n+1}\) is a collection of sets and 
\(f_1, f_2, \dots, f_n \) is a collection of functions such that \(f_i: A_i \rightarrow A_{i+1}\)
for all \(i \in \{1, 2, \dots, n \}\).

\emph{Let \(\circ\) denote the composition of functions so that 
\((g \circ h)(x) = g(h(x))\)}.

\begin{enumerate}
    \item {
        Explain why \(f_1 \circ f_2 \) is not well-defined.

        For a composition \(f_a \circ f_b\) to be well-defined, the codomain of 
        \(f_b\) must be the same as the domain of \(f_a\). We see that 
        \(f_2: A_2 \rightarrow A_3\), so the codomain of \(f_2\) is \(A_3\). 
        However, we see that \(f_1: A_1 \rightarrow A_2\), so the domain of 
        \(f_1\) is \(A_1\). Since \(A_1 \not= A_3\), \(f_1 \circ f_2\) is not 
        well-defined. 
    }

    \item {
        Explain why \(f_2 \circ f_1 \) is well-defined.

        We see that \(f_1: A_1 \rightarrow A_2\) and \(f_2: A_2 \rightarrow A_3\).
        The codomain of \(f_1\) is \(A_2\), which is the same as the domain of 
        \(f_2\). Thus, \(f_2 \circ f_1\) is well-defined. 
    }

    \item {
        Prove that \(f_n \circ f_{n-1} \circ \cdots \circ f_2 \circ f_1\) is a
        well-defined function, and state its domain and codomain. 

        \begin{proof}
            We proceed via induction. As our base case, we see that \(f_2 \circ f_1\)
            is well-defined. 

            For our inductive hypothesis, we assume that \(f_n \circ f_{n-1} \circ \cdots \circ f_2 \circ f_1\)
            is well-defined for \(n \le k\).

            Because \(g = f_k \circ f_{k-1} \circ \cdots \circ f_2 \circ f_1\) is 
            well-defined, and the codomain of \(f_k\) is \(A_{k+1}\), we see that 
            \(f_{k+1} \circ g\) is also well-defined. 

            Thus, by the principle of mathematical induction, we see that 
            \(f_n \circ f_{n-1} \circ \cdots \circ f_2 \circ f_1\) is 
            well-defined. 
        \end{proof}

        We see that because \(f_i : A_i \rightarrow A_{i+1}\), the domain of 
        \(f_1\) is \(A_1\), and the codomain of \(f_n\) is \(A_{n+1}\). Thus,
        the domain of our entire composed function is \(A_1\) and the codomain is 
        \(A_{n+1}\). 
    }
\end{enumerate}

\section*{Required Exercise 3 [4]}

\begin{enumerate}
    \item \begin{enumerate}
        \item {
            Three examples of integers that are in \(3 \mathbb{Z} + 2\) would 
            be 2, 5, and 8.
        }
        \item {
            The equivalence class an integer \(n\) belongs to is determined by 
            its \(\pmod 3\) value. 

            Integer \(n\) belongs to \(3\mathbb{Z} + j\)  if \(3 \equiv j \pmod 3\)
            for \(j \in \{0, 1, 2\}\).
        }
        \item {
            I know that every number is in exactly one of these equivalence 
            classes because 1) the nature of modular arithmetic, and 2) the 
            definition of an equivalence relation. 

            1) In modular arithmetic, each value maps to exactly one value 
            for any mod value we chose. 

            2) Equivalence are transitive. Now, if we assume that 
            an element is simmultaneously in two equivalence classes \(A\) 
            and \(B\), we run into a contradiction since all elements in 
            class \(A\) would now also be in class \(B\). Thus, \(A\) and 
            \(B\) are the same equivalence class. 
        }
    \end{enumerate}

    \item {
        Now let \(X = \{1, 2, 3, 4\}\) and let 
        \[R= \{(1,1), (1,2), (1,4), (2,1), (2,2), (2,4), (3,3), (4,1), (4,2), (4,4)\},\]
        which is to say that for \(a, b \in X, aRb\) if and only if \((a,b) \in R\). 
        \begin{itemize}
            \item {
                Prove that \(R\) is an equivalence relation. 

                \begin{proof}
                    For a relation to be an equivalence relation, it must be 
                    reflexive, symmetric, and transitive. 

                    We see that \(R\) is reflexive since for all \(a \in X\), 
                    \((a, a) \in R\). That is, all elements are related to themselves. 

                    We also see that \(R\) is symmetric because for all \((a, b) \in R\),
                    \((b, a) \in R\) where \(a, b \in X\). 

                    Lastly, we see that \(R\) is transitive because for all 
                    \(a, b, c \in X\), if \((a, b) \in R\) and \((b, c) \in R\),
                    \((a, c) \in R\). 

                    Thus, \(R\) is an equivalence relation. 
                \end{proof}
            }

            \item {
                Describe the two equivalence classes of \(X\). 

                \(3\) is in a class of its own, while \(\{1, 2, 4\} \) are in 
                another class. 
            }
        \end{itemize}
    }

    \item {
        \begin{itemize}
            \item {
                Prove that if \(R_1\) is not reflexive, then \(R_1\) does not 
                partition \(X\) into equivalence classes. 

                \begin{proof}
                    We see that for a relation to partition a set into equivalence
                    classes, all elements must belong to an equivalence class. 
                    However, since \(R_1\) is not reflexive, we see that there 
                    may be an element that does not belong to its own 
                    equivalence class. This is a contradiction, thus \(R_1\) can 
                    not partition \(X\) into equivalence classes.
                \end{proof}
            }

            \item {
                Prove that if \(R_2\) is not symmetric, then \(R_2\) does not 
                partition \(X\) into equivalence classes. 

                \begin{proof}
                    We proceed via contradiction by assuming that \(R_2\) is 
                    not symmetric and that it does partition \(X\) into 
                    equivalence classes. Since \(R_2\) is not symmetric, there 
                    must be an element \((a, b) \in R\) where \((b, a) \not \in R\)
                    for \(a, b \in X\). This means that there is an element \(a\) 
                    that is in the same class as \(b\), but \(b\) is not in the 
                    same class as \(a\). This is a contradiction, so \(R_2\) must 
                    not partition \(X\) into equivalence classes. 
                \end{proof}
            }

            \item {
                Prove that if \(R_3\) is not transitive, then \(R_3\) does not 
                partition \(X\) into equivalence classes. 

                \begin{proof}
                    We proceed via contradiction by assuming that \(R_3\) is 
                    not transitive and that it does partition \(X\) into 
                    equivalence classes. Since \(R_3\) is not transitive, there 
                    are elements \(a, b, c \in X\) where \(aRb\) and \(bRc\), but 
                    \(\neg aRc\). This means that \(a\) and \(b\) are in the 
                    same class, and \(b\) and \(c\) are the same class. But 
                    \(a\) and \(c\) are not in the same class. This is a 
                    contradiction, so \(R_3\) must not partition \(X \) into 
                    equivalence classes. 
                \end{proof}
            }
        \end{itemize}
    }
\end{enumerate}

\section*{Choice Exercise 8 [6]}

\begin{enumerate}
    \item {
        The sum of a rational number and an irrationl number is irrational. 

        \begin{proof}
            We proceed by contradiction. Assume that the sum of a rational 
            number \(a\) and an irrational number \(b\) is rational. Thus,
            \(a + b = c/d\) for \(c, d \in \mathbb{N }\).

            Now, we see that since \(c \in \mathbb{N}\), \(c = c_1 + c_2\) for 
            some \(c_1, c_2 \in \mathbb{N}\). Thus, 
            \begin{align*}
                a + b &= \frac{c }{d} \\
                &= \frac{c_1 + c_2 }{d} \\
                &= \frac{c_1}{d } + \frac{c_2 }{d }.
            \end{align*}

            This is a contradiction because it shows that both \(a\) and \(b\) 
            are rational. 
            Thus, the sum of a rational number and an irrationl number is irrational. 
        \end{proof}

        \begin{proof}
            We proceed by contrapositive. First, we see that or proposition can 
            be reworded as ``If \(a\) is rational and \(b \) is irrational, then 
            \(a+b \) is irrational.''

            Thus, the contrapositive of this statement is ``If \(a + b \) is 
            rational, then \(a \) is irrational or \(b\) is rational.''

            First, we see that \(a + b = c/d \) for some \(c, d \in \mathbb{N}\).
            Since \(c \in \mathbb{N}\), \(c = c_1 + c_2\) for some \(c_1, c_2 \in \mathbb{N}\).
            Thus,
            \begin{align*}
                a + b &= \frac{c }{d} \\
                &= \frac{c_1 + c_2 }{d} \\
                &= \frac{c_1}{d } + \frac{c_2 }{d }.
            \end{align*}

            We now see that \(b \) is rational.
            Thus, the sum of a rational number and an irrationl number is irrational. 
        \end{proof}
    }

    \item {
        Suppose \(a, b \) and \(c \) are positive real numbers. If \(ab = c \)
        then \(a \le \sqrt{c }\) or \(b \le \sqrt{c}\).

        \begin{proof}
            We proceed via contradiction. Assume that \(ab = c\) and \(a > \sqrt{c}\)
            and \(b > \sqrt{c}\). 

            Since \(a > \sqrt{c}\) and \(b > \sqrt{c}\), \(ab > \sqrt{c} \sqrt{c} = c\).
            
            This is a contradiction, so \(ab = c \) implies \(a \le \sqrt{c }\)
            or \(b \le \sqrt{c}\).
        \end{proof}

        \begin{proof}
            We proceed via contrapositive. We reword our proposition as ``If 
            \(a > \sqrt{c}\) and \(b > \sqrt{c}\), then \(ab \not= c\).''

            Since \(a > \sqrt{c}\) and \(b > \sqrt{c}\), \(ab > \sqrt{c}\sqrt{c} = c\).
            Thus, \(ab \not= c\). So, via contrapositive, we see \(ab = c\) implies 
            \(a \le \sqrt{c }\) and \(b \le \sqrt{c }\).
        \end{proof}
    }

    \item {
        Suppose that \(n \) is a composite integer, then there exists a prime divisor 
        of \(n \) that is less than or equal to \(\sqrt{n }\).

        \begin{proof}
            We proceed via contradiction. Assume that \(n \) is a composite 
            integer, and there does not exist a prime divisor of \(n \) that is 
            less than or equal to \(\sqrt{n}\). 

            Notice that for all integers \(a, b > \sqrt{n}\), \(ab > n\). Thus,
            there must be no prime divisors of \(n \), meaning \(n\) is prime. 

            This is a contradiction, so there exists a prime divisor 
            of \(n \) that is less than or equal to \(\sqrt{n }\).
        \end{proof}

        \begin{proof}
            We proceed via contrapositive. We reword our proposition as ``If 
            there does not exist a prime divisor of \(n \) that is less than or 
            equal to \(\sqrt{n } \), then \(n \) is prime.''

            Notice that for all integers \(a, b > \sqrt{n}\), \(ab > n\). Thus, 
            there can not exist any prime divisors of \(n \) greator than \(\sqrt{n}\).

            Also, because there does not exist a prime divisor of \(n \) that is 
            less than or equal to \(\sqrt{n}\), there does not exist any prime 
            divisors of \(n \). Thus, \(n\) is prime.
        \end{proof}
    }
\end{enumerate}

\section*{Choice Exercise 10 [8]}

\begin{enumerate}
    \item {
        [1] Compute by hand the number of relations on a set \(X\) where \(|X| = n\).

        All relations are a subset of the \(X \times X\). Thus, the number of 
        possible relations is the cardinality of the power set of \(X^2\). 
        Thus, the number of possible relations is \(2^{n^2}\).
    }
    \item {
        [2] Compute by hand the number of relations that are symmetric but 
        not reflexive on a set \(X\) where \(|X| = n\). 

        We first compute the number of relations that are symmetrical. We do this 
        be taking the noticing that this essentially takes the ``upper triangle''
        of our cartesian product because we must include the ``other half''. 

        We get the formula as \(n^2 - (n-1)^2 + (n-2)^2 \dots = \sum_{i=0}^{n}(n-i)^2 (-1)^i\).
        Now, we find the cardinality of the powerset of a set with said number of 
        elements. Thus, we have the formula \(2^{\sum_{i=0}^{n}(n-i)^2 (-1)^i}\).
        I later found out that the formula for the number of elements in the 
        ``upper triangle'' is actually just \(\frac{n(n+1)}{2}\). So, a simpler
        formula would be \(2^{\frac{n(n+1)}{2}}\).

        Now, the number of symmetric and reflexive relations is sampled from 
        the ``lower triangle'' because the diagonal is implicitly selected.
        The formula for this is \(\frac{n(n-1)}{2}\). Similarly, all possible 
        relations would be \(2^{\frac{n(n-1)}{2}}\).

        Finally, we get \(2^{\frac{n(n+1)}{2}} - 2^{\frac{n(n-1)}{2}}\) as 
        the formula for the number of relations on set \(X\) that is 
        symmetric and not reflexive. 
    }

    \item {
        [3] Write a program to count the number of relations that are transitive
        on a set \(X\) where \(|X| = n\). Confirm that \(f(4) = 3994\). 

        \lstinputlisting[language=Python]{./code/num_transitive.py}
    }

    \item {
        [2] We say a relation is antitransitive if \(xRy\) and \(yRz\) implies 
        \(\sim(xRz)\). By any means possible determine the number of antitransitive 
        relations on  \(X\) when \(|X| = 5\). 

        To save you from running the code (takes a couple minutes), 
        the final result is 471552.

        \lstinputlisting[language=Python]{./code/num_antitransitive.py}
    }


\end{enumerate}

\includepdf[pages=-]{portfolio.pdf}

\end{document}