\documentclass{article}
\usepackage[margin=1in]{geometry}
\usepackage{mathtools, amsfonts, amsthm, amssymb, graphicx, listings, xcolor, pdfpages}

\newtheorem{proposition}{Proposition}

\definecolor{codegreen}{rgb}{0,0.6,0}
\definecolor{codegray}{rgb}{0.5,0.5,0.5}
\definecolor{codepurple}{rgb}{0.58,0,0.82}
\definecolor{backcolour}{rgb}{0.95,0.95,0.92}

\lstdefinestyle{mystyle}{
    backgroundcolor=\color{backcolour},   
    commentstyle=\color{codegreen},
    keywordstyle=\color{magenta},
    numberstyle=\tiny\color{codegray},
    stringstyle=\color{codepurple},
    basicstyle=\ttfamily\footnotesize,
    breakatwhitespace=false,         
    breaklines=true,                 
    captionpos=b,                    
    keepspaces=true,                 
    numbers=left,                    
    numbersep=5pt,                  
    showspaces=false,                
    showstringspaces=false,
    showtabs=false,                  
    tabsize=2
}

\lstset{style=mystyle}

\title{HW10}
\author{Sam Ly}

\begin{document}
\maketitle

\section*{Total points: 21}

\section*{Required Exercise 1 [3]}
\begin{enumerate}
    \item {
        Done. HW10 Exercise 1: My favorite choice exercise was all the way back in HW02. Specifically choice exercise 5.3. Hint: ||Think about how to translate the "concat" function as a multiplication.||
    }
    \item {
        HW09 Exercise 5

        \begin{enumerate}
            \item[1.] {
                [2] Suppose that \(A\) and \(B\) are finite sets and \(|A| = |B|\). Prove that a function \(f \colon A \to B\) is injective if and only if it is surjective.

                \begin{proof}
                    First, we assume that \(f\) is injective. Since \(|A| = |B|\), 
                    by the Pigeonhole Principle, \(f\) must also be surjective because 
                    every unique element in \(A\) is mapped to a unique element in \(B\). 
                    Then, we assuming that \(f\) is surjective. Since \(|A| = |B|\), 
                    for every \(b \in B\), there exists one and only one \(a \in A\) 
                    such that \(f(a) = b\). Thus, \(f\) is injective. 

                    Therefore, \(f\) is injective if and only if \(f\) is surjective. 
                \end{proof}
            }

            \item[2.] {
                [2] Suppose that \(A\) and \(B\) are finite sets where \(|A| = n\) and \(|B| = m\). Determine the number of functions \(f \colon A \to B\).

                Each element in the domain can be mapped to any element in the codomain. 
                Thus, for the \(n\) elements in the domain, there can be \(m\) elements 
                that \(f(a)\) maps to. Thus, the total number of functions \(f: A \rightarrow B\) is 
                \(m^n\).
            }
        \end{enumerate}
    }
\end{enumerate}

\section*{Required Exercise 2 [7]}
\begin{enumerate}
    \item Done.
    \item Done.
    \item Done.
\end{enumerate}

\section*{Choice Exercise 6 [4]}
Before proceding to the actual problem, I would like to mention that I reused 
the code that takes the cartesian product of two infinite sequences since it 
uses the same antidiagonal technique.
\lstinputlisting[language=Python, lastline=28]{./code/countably_inf.py}

\begin{enumerate}
    \item {
        Write a program that prints out the first \(50\) terms of the sequence, \(a(0)\) through \(a(49)\).

        \lstinputlisting[language=Python, firstline=30, lastline=37]{./code/countably_inf.py}

        \begin{verbatim}
0: 0
1: 1
2: 1
3: 2
...
49: 1
        \end{verbatim}
    }

    \item {
        Write a program that prints out the first \(5151st\) term of the sequence, \(a(5150)\).

        \lstinputlisting[language=Python, firstline=39, lastline=41]{./code/countably_inf.py}
        \begin{verbatim}
5151: 0
        \end{verbatim}
    }

    \item {
        Write a program that prints out the \(m\)-th term of the sequence, but without computing any smaller term. In other words, given some integer \(m\), figure out how to determine \(n\) and \(k\) such that \(a(m) = T(n,k)\), and then use this to compute \(a(m)\).

        \lstinputlisting[language=Python, firstline=42]{./code/countably_inf.py}
    }
\end{enumerate}

\section*{Choice Exercise 7 [4]}

\begin{enumerate}
    \item[2.] {
        In abstract algebra, we can define the rational numbers in the following way.
    
        \begin{enumerate}
            \item {
                We define a relation on the set \(X = \mathbb{Z} \times \left(\mathbb{Z} \setminus \{0\} \right)\) 
                by saying that \((a, b) \sim (a', b')\) if and only if \(ab' = a'b\). 
                \begin{enumerate}
                    \item {
                        Prove that this relation is reflexive.

                        \begin{proof}
                            Intuitively, we see that the relation iterally has a ``=''
                            in it, so it is probably an equivalence relation. 

                            First, we see that if \((a, b) = (a', b')\) if and 
                            only if \(a = a'\) and \(b = b'\). Thus, \(ab' = a'b = ab\),
                            so \(\sim\) is reflexive.
                        \end{proof}
                    }
                    \item {
                        Prove that this relation is symmetric.

                        \begin{proof}
                            We see that if \((a, b) \sim (a', b')\), then \(ab' = a'b\).
                            Also, \((ab' = a'b) \Leftrightarrow (a'b = ab')\). Thus, 
                            \((a', b') \sim (a, b)\). Therefore, \(\sim\) is symmetric. 
                        \end{proof}
                    }
                    \item {
                        Prove that this relation is transitive.

                        \begin{proof}
                            Let \(a_0, b_0), (a_1, b_1), (a_2, b_2)\) such that 
                            \((a_0, b_0) \sim (a_1, b_1)\) and \((a_1, b_1) \sim (a_2, b_2)\).

                            Thus, \(a_0b_1 = a_1b_0\) and \(a_1b_2 = a_2b_1\). 

                            We see 
                            \begin{align*}
                                b_1 &= \frac{a_1b_2}{a_2}, \\
                                a_1 &= \frac{a_2b_1}{b_2}, \\
                                \text{and } \frac{a_1}{a_2} &= \frac{b_1}{b_2}.
                            \end{align*}

                            By substituting in \(b_1\) and \(a_1\), we get
                            \begin{align*}
                                a_0 \frac{a_1b_2}{a_2} &= b_0 \frac{a_2b_1}{b_2}.
                            \end{align*}

                            Since \(\frac{a_1}{a_2} = \frac{b_1}{b_2}\), they can 
                            be cancelled out, resulting in \(a_0b_2 = a_2b_0\). 
                            Thus, \((a_0, b_0) \sim (a_2, b_2)\), and \(\sim\) 
                            is transitive.
                        \end{proof}
                    }
                \end{enumerate}
            }
            \item {
                Now we introduce new notation: we write \(\mathbb{Q} = X/\sim\), where \(X/\sim\) means that we count elements that are related by \(\sim\) as "the same". To explain why this makes sense, show that \(\frac{a}{b} = \frac{a'}{b'}\) if and only if \((a, b) \sim (a', b')\).

                \begin{proof}
                    By definition, \((a, b) \sim (a', b')\) if and only if \(ab' = a'b\).
                    Thus, \((a, b) \sim (a', b')\) if and only if \(\frac{a }{b } = \frac{a' }{b' }\).
                \end{proof}
            }
            \item {
                Show that defining \((a, b) + (a', b') = (ab' + a'b, bb')\) is 
                analogous to how we define \(\frac{a}{b} + \frac{a'}{b'}\).

                We define \(\frac{a}{b} + \frac{a'}{b'}\) as
                \begin{align*}
                    \frac{a }{b } + \frac{a'}{b' } = \frac{a }{b } \times \frac{b'}{b'} + \frac{a'}{b'} \times \frac{b }{b } = \frac{ab'}{bb'} + \frac{a'b }{bb'} = \frac{ab' + a'b }{bb'}.
                \end{align*}

                Then, we see that the for our tuples \(x, y\), \(x\) can be seen 
                as the numerator and \(y\) can be seen as the denominator. 

                Thus, \((a, b) + (a', b') = (ab' + a'b, bb')\) is 
                analogous to how we define \(\frac{a}{b} + \frac{a'}{b'}\).
            }

            \item {
                Show that defining \((a, b) \times (a', b') = (aa', bb')\) is analogous to how we define \(\frac{a}{b} \times \frac{a'}{b'}\).

                Using similar logic as the previous problem, we see that 
                \[\frac{a }{b } \times \frac{a'}{b'} = \frac{aa'}{bb'},\]
                so defining \((a, b) \times (a', b') = (aa', bb')\) is analogous to how we define \(\frac{a}{b} \times \frac{a'}{b'}\).
            }
        \end{enumerate}
    }
\end{enumerate}

\section*{Choice Exercise 9 [3]}

\begin{enumerate}
    \item {
        Recall that given a set \(A \), the power set \(\mathcal{P}(A)\) is the 
        set of all of its subsets. Write down the eight subsets of \(\{1, 2, 3\}\). 

        \[\mathcal{P}(\{1, 2, 3\} = \{\emptyset, \{1\}, \{2\}, \{3\}, \{1,2\}, \{1,3\}, \{2,3\}, \{1,2,3\}\}\]
    }

    \item {
        Find the error in the following false proof. 
        \begin{proposition}
            There exists an injection \(f: \mathcal{P }(\mathbb{N }) \rightarrow \mathbb{N} \). 
        \end{proposition}
        \begin{proof}
            Let \(p_i\) be the \(i\)-th prime, so \(p_0 = 2\), \(p_1 = 3\), 
            \(p_2 = 5\), \(p_3 = 7\), \(p_4 = 11\), and so on. Then define 
            \(f \colon \mathcal{P}(\mathbb{N}) \to \mathbb{N}\) as 
            
            \[f(\{a_1, a_2, \dots, a_n\}) = p_{a_1}p_{a_2} \cdots p_{a_n},\]

            where \(f(\emptyset) = 1\).
            (For example, \(f(\{0, 2, 4\}) = p_0p_2p_4 = 2 \cdot 5 \cdot 11 = 110\).)
    
            Because each distinct element \(S \in \mathcal{P}(\mathbb{N})\) maps 
            to a number with a distinct prime factorization, \(f(S_1) = f(S_2)\) 
            if and only if \(S_1 = S_2\), therefore \(f\) is injective (1-to-1).
        \end{proof}

        The error in this proof is that it doesn't factor in the infinite subsets 
        of \(\mathbb{N}\). This is because the product of an infinite subset of 
        \(\mathbb{N }\) is not defined in \(\mathbb{N}\). For example, \(\mathbb{N}\)
        is a subset of \(\mathbb{N}\), so is in \(\mathcal{P}(\mathbb{N})\). However,
        we are not able to multiply together an infinite number of natural numbers 
        to find a natural number. Thus, \(f(\mathbb{N})\) is not defined, and so 
        \(f\) is not an injection from \(\mathcal{P}(\mathbb{N})\) to \(\mathbb{N}\).
    }

    \item {
        If there were such an injection, it would prove that the cardinality of \(\mathcal{P}(\mathbb{N})\) is less than or equal to the cardinality of \(\mathbb{N}\), and thus \(\mathcal{P}(\mathbb{N})\) is countably infinite. However, it turns out that the cardinality of \(\mathcal{P}(\mathbb{N})\) is the same as the cardinality of \(\mathbb{R}\). Write a sentence or two about any thoughts, questions, or observations about this.

        One observation about this is that set of decimal numbers of length 
        at most \(n\) can be thought of as the cartesian product of 
        \(A = \{0, 1, 2, 3, 4, 5, 6, 7, 8, 9\}\) \(n\)
        times. For example, the set of all decimal numbers of length at most 3 is 
        equal to \(A \times A \times A = A^3\). So, let \(\mathbb{D}_n\) be the 
        set of decimal numbers of length at most \(n\). There exists a bijection 
        \(f: \mathbb{D}_n \rightarrow A^n\) for all \(n \in \mathbb{N}\). 

        Now, if we generalize this to \(\mathbb{R}\), we see that \(\mathbb{R}\)
        has numbers with infinitely long decimal expansions. If we were to assume 
        (I'm not sure of the truth of this statement) that even the longest 
        numbers in \(\mathbb{R}\) have a countably infinite number of digits, 
        then \(\mathbb{R}\) must have the same cardinality as the cartestian product of 
        \(A\) a countably infinite number of times. Recall that the cartesian product 
        of finite sets a finite number of times will always be a finite set. Similarly, 
        the cartesian product of countably infinite sets a finite number of times 
        will also be countable infinite. However, taking the cartesian product of 
        finite or countably infinite sets a \emph{countably infinite} number of times 
        results in an \emph{uncountably infinite} set. This leads me to wonder if 
        more formal algebras exist to describe ``higher order inifinities.''

        We then see that the power set of a countably infinite set must be 
        uncountably infinite because the \(\mathcal{P }(S) = \{0, 1\}^S\). This 
        formula can be thought of as ``all combinations of including or not including 
        a specific element of \(S\).'' Thus, the cardinality of \(\mathcal{P }(\mathbb{N})\) 
        is equal to the cardinality of \(\{0, 1\}\) (a finite set) times itself 
        a countably infinite number of times, which is uncountably infinite.
    }
\end{enumerate}


\includepdf[pages=-]{portfolio.pdf}
\end{document}