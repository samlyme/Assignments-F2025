\documentclass{article}
\usepackage[margin=1in]{geometry}
\usepackage{mathtools, amsfonts, amsthm}

\title{HW 1}
\author{Sam Ly}

\begin{document}
\maketitle

\section*{[4 pts] Required Exercise 2.}

\begin{enumerate}
    \item {
        If the two numbers \(a_1\) and \(a_2\) are odd numbers,
        then their product \(a_1 \times a_2\) is an odd number.

        \begin{proof}
            By definition, \(a_1 = 2n_1 + 1\) and \(a_2 = 2n_2 + 1\).

            Then, 
            \[ a_1 \times a_2 = (2n_1 + 1) \times (2n_2 + 1)\]
            \[ = 4n_1n_2 + 2n_1 + 2n_2 + 1\]
            \[ = 2(2n_1n_2 + n_1 + n_2) + 1 \]

            Thus, since we can write \(a_1 \times a_2 \) in the form \(2k + 1 \) for some
            integer \(k\), we prove that \(a_1 \times a_2 \) is odd.
        \end{proof}
    }

    \item {
        2. If three numbers $a_1$, $a_2$, and $a_3$ are odd numbers, then their product 
        \(a_1 \times a_2 \times a_3 \) is an odd number.

        \begin{proof}
            From the previous proof, we can substitute \(a_1 \times a_2 \) for 
            some odd integer \(x\).

            Now we have \(x \times a_3\), where both \(x\) and \(a_3\) are odd.
            Again, using the previous proof, we know \(x \times a_3\).

            Thus, \(a_1 \times a_2 \times a_3 \) is odd.
        \end{proof}
    }

    \item {
        If the four numbers \(a_1, a_2, ..., a_4 \)
        are odd numbers, then their product \(a_1 \times a_2 \times ... \times a_4\)
        is an odd number.

        \begin{proof}
            Using the first proof, we know \(a_1 \times a_2 \)  and \(a_3 \times a_4\) 
            are both odd.

            Thus, \((a_1 \times a_2) \times (a_3 \times a_4)\) is odd.

            By the associative property of multiplication,
            \[a_1 \times a_2 \times ... \times a_4 = (a_1 \times a_2) \times (a_3 \times a_4), \]
            and therefore \(a_1 \times a_2 \times ... \times a_4 \) is odd.
        \end{proof}
    }

    \item {
        If the 50 numbers \(a_1, a_2, ..., a_{50} \)
        are odd numbers, then their product \(a_1 \times a_2 \times ... \times a_50\)
        is an odd number.

        \begin{proof}
            Using our first proof, we define our base case as "the product of
            two odd integers is odd".

            Let \(S_k\) be the sequence of arbitrary odd integers \(a_1, a_2, ..., a_k\), 
            
            We assume that for \(S_k = a_1, a_2, ..., a_k\), the product 
            \(\prod{S_k} = a_1 \times a_2 \times ... \times a_k\) is odd.

            Let \(a_{k+1}\) be an odd integer.

            Then, the product of the sequence \(S_{k+1} = a_1, a_2, ..., a_k, a_{k+1}\)
            can be written as \(\prod{S_{k+1}} = \prod{S_k} \times a_{k+1} \).

            Since the product of two odd numbers is odd, and \(\prod{S_k}\) and \(a_{k+1}\) 
            are both odd, the product of the two is also odd. 

            Therefore, by induction, the product of any number of odd numbers will 
            always be odd. Thus, the product of \(50\) odd numbers is odd.
        \end{proof}
    }
\end{enumerate}

\section*{[2 pts] Required Exercise 3.}

3. Give a hint about one of the problems on the homework.

\section*{[5 pts] Choice Exercise 4.}

\begin{enumerate}
    \item \(S = \{1, 2, 3\}\)

    \item {
        First way:
        \[\{n \in \mathbb{N} | \log(n) > 5 \} \]

        Second way:
        \[\{n \in \mathbb{N} \mid \log(n) > 5 \}\]

        The main difference is that the use of the pipe symbol causes the rendering
        to look funky, since the bar is too close to the "N". Using the "mid" command 
        renders the bar with nicer spacing.
    }

    \item {
        With:
        \[ \left( \int e^x dx \right) \]

        Without:
        \[ (\int e^x dx) \]

        Not using the left and right descriptors causes the parens to not "cover"
        the entire height of the expression.
    }

    \item {
        \[\{n \in \mathbb{N} \mid n is even \}\]
        \[\{n \in \mathbb{N} \mid n \text{is even} \}\]
        \[\{n \in \mathbb{N} \mid n \text{ is even} \}\]

        The first option is not what we intend, since all letters in "is even"
        are treated as independent variable names, and gets rendered as such. 

        The second option is better, since "is even" is rendered properly, but 
        we are missing a space in the front of "is even", so the rendered text
        gets squished. 

        The third option is best and is likely what we intended to write.
    }

    \item {
        \[\big(\underbrace{1,1,\dots,1}_{k\text{ times}}\big)\]
        \[\left(\underbrace{1,1,\dots,1}_{k\text{ times}}\right)\]
        \[\underbrace{\left(1,1,\dots,1\right)}_{k\text{ times}}\]

        When compared to the first, the second option makes the parens
        look funny because they really don't need to be that tall. 

        The third option looks fine, but the underbrace extending past 
        the parens may have some weird "semantics".

        I would prefer to read the first option because it is (imo)
        sematntically correct, since we have \(k\) elements inside of
        the tuple. But writing it could be a bit cumbersome as we need to
        use the "big" command explicitly. When writting, the third option 
        is easiest to remember, and imo still reasonably readable.
    }
\end{enumerate}

\section*{[5 pts] Choice Exercise 5.}

\begin{enumerate}
    \item {
        Compute an integral to show that when \(z=1\), \(\Gamma(1) = 0!\).
        \[\Gamma(1) = \int_0^\infty{t^{1-1}e^{-t}} dt = \int_0^\infty{e^{-t}} dt = 1 = 0!\]
    }

    \item {
        Compute an integral to show that when \(z=2\), \(\Gamma(2)=1!\).
        \[\Gamma(2) = \int_0^\infty{te^{-t}} dt\]
        \[ = \left (-t^2 e^{-t}) \right|_0^\infty + \int_{0}^{\infty}{t e^{-t}} dt\]
        \[ = 0 + 1 = 1!\]
    }

    \item {
        Compute an integral to show that when \(z=3\), \(\Gamma(3)=2!\).

        \[ \Gamma(3) = \int_0^\infty{t^2e^{-t}} dt\]
        \[ = \left (-t^2e^{-t}) \right|_0^\infty + 2\int_{0}^{\infty}{te^{-t}} dt\]
        \[ = 0 + 2 = 2!\]
    }

    \item {
        Prove that when \(z \ge 1\) is an integer, \(\Gamma(z) = (z-1)\Gamma(z-1)\).

        \begin{proof}
            We actually found it easier to say that \(\Gamma(k+1) = k\Gamma(k)\) for
            some integer \(k \ge 1\). Theses two statements are equivalent.

            First, we define \(\Gamma(k)\):
            \[ \Gamma(k) = \int_{0}^{\infty}{t^{k-1}e^{-t}} dt\]

            Then, we define \(\Gamma(k+1)\):
            \[ \Gamma(k+1) = \int_{0}^{\infty}{t^ke^{-t}} dt\]
            \[ = \left (-t^ke^{-t}) \right|_0^\infty + \int_{0}^{\infty}{kt^{k-1}e^{-t}} dt\]
            \[ = 0 + k \int_{0}^{\infty}{t^{k-1}e^{-t}} dt\]

            Thus,
            \[ \Gamma(k+1) = k \int_{0}^{\infty}{t^{k-1}e^{-t}} dt = k\Gamma(k)\]
        \end{proof}
    }

    \item {
        Why can you conclude that \(\Gamma(z) = (z-1)!\) for all integers \(z \ge 1\)?

        We are able to conclude that \(\Gamma(z) = (z-1)!\) for all integers \(z \ge 1\)
        because we have actually proved this statement via induction.

        We have shown that for any integer \(k \ge 1\), if \(\Gamma(k) = (k-1)!\)
        then \(\Gamma(k+1) = k!\).

        We have also shown a few base cases for which the statement \(\Gamma(n) = (n-1)!\),
        namely for \(n = 1, 2, 3\).

        Thus, through induction, we have proved that \(\Gamma(z) = (z-1)!\) for all integers \(z \ge 1\).
    }
\end{enumerate}
\end{document}