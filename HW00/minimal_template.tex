\documentclass{article}
\usepackage[margin=1in]{geometry}
\usepackage{mathtools, amsfonts, amsthm}

\title{HW 0}
\author{Sam Ly}

\begin{document}
\maketitle

\section*{Exercise 2}
I already use Latex so it was easy :).

\section*{Exercise 3}
The following integral converges: 
\[
    \int_{x=\varepsilon}^\infty \frac{1}{x^2} dx = \frac{1}{\varepsilon}.
\]
This looks how I expected.
\\The following integral diverges: 
\[
    \int_1^\infty \frac 1 x dx .
\]

I have some weird indents in my pdf. I'm not sure why that is.

\section*{Exercise 4}

One of the most unexpected things about effective mathematical communication is 
the notion of writing math in \textbf{complete sentences}. I've also always viewed
mathematical shorthand as sophisticated (albeit extremely unreadable), but this 
guideline really vindicates my prior feelings about it. I can finally just write
my thoughts without feeling like I have to finaggle them into some arbitrary
symbols, instead of just writing them in natural language. Other than that, I
think this will be a fun course!


\section*{Exercise from class}

\begin{proof}
    Since \(a\) is odd, we can write $a$ as \(a = 2m + 1\) for some integer \(m\).
    Likewise, we can write \(b = 2k + 1\) for some integer \(k\). 
    \\ Therefore,
    \[
        a + b = 2m + 1 + 2k + 1 = 2(m + k + 1),
    \]
    where \(m + k + 1\) is an integer, since adding integers results in an integer.
    Thus \(a + b\) is even.

\end{proof}

\end{document}